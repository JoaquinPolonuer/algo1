\documentclass[a4paper]{article}

\setlength{\parskip}{2mm}
\newcommand{\tab}{~ \qquad}
\usepackage{ifthen}
\usepackage{amssymb}
\usepackage{multicol}
\usepackage{graphicx}
\usepackage[absolute]{textpos}
\usepackage{amsmath, amscd, amssymb, amsthm, latexsym}
% \usepackage[noload]{qtree}
%\usepackage{xspace,rotating,calligra,dsfont,ifthen}
\usepackage{xspace,rotating,dsfont,ifthen}
\usepackage[spanish,activeacute]{babel}
\usepackage[utf8]{inputenc}
\usepackage{pgfpages}
\usepackage{pgf,pgfarrows,pgfnodes,pgfautomata,pgfheaps,xspace,dsfont}
\usepackage{listings}
\usepackage{multicol}
\usepackage{todonotes}
\usepackage{url}
\usepackage{float}
\usepackage{framed,mdframed}
\usepackage{cancel}

\usepackage[strict]{changepage}


\makeatletter


\newcommand\hfrac[2]{\genfrac{}{}{0pt}{}{#1}{#2}} %\hfrac{}{} es un \frac sin la linea del medio

\newcommand\Wider[2][3em]{% \Wider[3em]{} reduce los m\'argenes
\makebox[\linewidth][c]{%
  \begin{minipage}{\dimexpr\textwidth+#1\relax}
  \raggedright#2
  \end{minipage}%
  }%
}


\@ifclassloaded{beamer}{%
  \newcommand{\tocarEspacios}{%
    \addtolength{\leftskip}{4em}%
    \addtolength{\parindent}{-3em}%
  }%
}
{%
  \usepackage[top=1cm,bottom=2cm,left=1cm,right=1cm]{geometry}%
  \usepackage{color}%
  \newcommand{\tocarEspacios}{%
    \addtolength{\leftskip}{3em}%
    \setlength{\parindent}{0em}%
  }%
}

\newcommand{\encabezadoDeProc}[4]{%
  % Ponemos la palabrita problema en tt
%  \noindent%
  {\normalfont\bfseries\ttfamily proc}%
  % Ponemos el nombre del problema
  \ %
  {\normalfont\ttfamily #2}%
  \
  % Ponemos los parametros
  (#3)%
  \ifthenelse{\equal{#4}{}}{}{%
  \ =\ %
  % Ponemos el nombre del resultado
  {\normalfont\ttfamily #1}%
  % Por ultimo, va el tipo del resultado
  \ : #4}
}

\newcommand{\encabezadoDeTipo}[2]{%
  % Ponemos la palabrita tipo en tt
  {\normalfont\bfseries\ttfamily tipo}%
  % Ponemos el nombre del tipo
  \ %
  {\normalfont\ttfamily #2}%
  \ifthenelse{\equal{#1}{}}{}{$\langle$#1$\rangle$}
}

% Primero definiciones de cosas al estilo title, author, date

\def\materia#1{\gdef\@materia{#1}}
\def\@materia{No especifi\'o la materia}
\def\lamateria{\@materia}

\def\cuatrimestre#1{\gdef\@cuatrimestre{#1}}
\def\@cuatrimestre{No especifi\'o el cuatrimestre}
\def\elcuatrimestre{\@cuatrimestre}

\def\anio#1{\gdef\@anio{#1}}
\def\@anio{No especifi\'o el anio}
\def\elanio{\@anio}

\def\fecha#1{\gdef\@fecha{#1}}
\def\@fecha{\today}
\def\lafecha{\@fecha}

\def\nombre#1{\gdef\@nombre{#1}}
\def\@nombre{No especific'o el nombre}
\def\elnombre{\@nombre}

\def\practicas#1{\gdef\@practica{#1}}
\def\@practica{No especifi\'o el n\'umero de pr\'actica}
\def\lapractica{\@practica}


% Esta macro convierte el numero de cuatrimestre a palabras
\newcommand{\cuatrimestreLindo}{
  \ifthenelse{\equal{\elcuatrimestre}{1}}
  {Primer cuatrimestre}
  {\ifthenelse{\equal{\elcuatrimestre}{2}}
  {Segundo cuatrimestre}
  {Verano}}
}


\newcommand{\depto}{{UBA -- Facultad de Ciencias Exactas y Naturales --
      Departamento de Computaci\'on}}

\newcommand{\titulopractica}{
  \centerline{\depto}
  \vspace{1ex}
  \centerline{{\Large\lamateria}}
  \vspace{0.5ex}
  \centerline{\cuatrimestreLindo de \elanio}
  \vspace{2ex}
  \centerline{{\huge Pr\'actica \lapractica -- \elnombre}}
  \vspace{5ex}
  \arreglarincisos
  \newcounter{ejercicio}
  \newenvironment{ejercicio}{\stepcounter{ejercicio}\textbf{Ejercicio
      \theejercicio}%
    \renewcommand\@currentlabel{\theejercicio}%
  }{\vspace{0.2cm}}
}


\newcommand{\titulotp}{
  \centerline{\depto}
  \vspace{1ex}
  \centerline{{\Large\lamateria}}
  \vspace{0.5ex}
  \centerline{\cuatrimestreLindo de \elanio}
  \vspace{0.5ex}
  \centerline{\lafecha}
  \vspace{2ex}
  \centerline{{\huge\elnombre}}
  \vspace{5ex}
}


%practicas
\newcommand{\practica}[2]{%
    \title{Pr\'actica #1 \\ #2}
    \author{Algoritmos y Estructuras de Datos I}
    \date{Primer Cuatrimestre 2022}

    \maketitlepractica{#1}{#2}
}

\newcommand \maketitlepractica[2] {%
\begin{center}
\begin{tabular}{r cr}
 \begin{tabular}{c}
{\large\bf\textsf{\ Algoritmos y Estructuras de Datos I\ }}\\
Primer Cuatrimestre 2022\\
\title{\normalsize Gu\'ia Pr\'actica #1 \\ \textbf{#2}}\\
\@title
\end{tabular} &
\begin{tabular}{@{} p{1.6cm} @{}}
\includegraphics[width=1.6cm]{logodpt.jpg}
\end{tabular} &
\begin{tabular}{l @{}}
 \emph{Departamento de Computaci\'on} \\
 \emph{Facultad de Ciencias Exactas y Naturales} \\
 \emph{Universidad de Buenos Aires} \\
\end{tabular}
\end{tabular}
\end{center}

\bigskip
}


% Símbolos varios

\newcommand{\nat}{\ensuremath{\mathds{N}}}
\newcommand{\ent}{\ensuremath{\mathds{Z}}}
\newcommand{\float}{\ensuremath{\mathds{R}}}
\newcommand{\bool}{\ensuremath{\mathsf{Bool}}}
\newcommand{\True}{\ensuremath{\mathrm{true}}}
\newcommand{\False}{\ensuremath{\mathrm{false}}}
\newcommand{\Then}{\ensuremath{\rightarrow}}
\newcommand{\Iff}{\ensuremath{\leftrightarrow}}
\newcommand{\implica}{\ensuremath{\longrightarrow}}
\newcommand{\IfThenElse}[3]{\ensuremath{\mathsf{if}\ #1\ \mathsf{then}\ #2\ \mathsf{else}\ #3\ \mathsf{fi}}}
\newcommand{\In}{\textsf{in }}
\newcommand{\Out}{\textsf{out }}
\newcommand{\Inout}{\textsf{inout }}
\newcommand{\yLuego}{\land _L}
\newcommand{\oLuego}{\lor _L}
\newcommand{\implicaLuego}{\implica _L}
\newcommand{\cuantificador}[5]{%
	\ensuremath{(#2 #3: #4)\ (%
		\ifthenelse{\equal{#1}{unalinea}}{
			#5
		}{
			$ % exiting math mode
			\begin{adjustwidth}{+2em}{}
			$#5$%
			\end{adjustwidth}%
			$ % entering math mode
		}
	)}
}

\newcommand{\existe}[4][]{%
	\cuantificador{#1}{\exists}{#2}{#3}{#4}
}
\newcommand{\paraTodo}[4][]{%
	\cuantificador{#1}{\forall}{#2}{#3}{#4}
}

% Símbolo para marcar los ejercicios importantes (estrellita)
\newcommand\importante{\raisebox{0.5pt}{\ensuremath{\bigstar}}}


\newcommand{\rango}[2]{[#1\twodots#2]}
\newcommand{\comp}[2]{[\,#1\,|\,#2\,]}

\newcommand{\rangoac}[2]{(#1\twodots#2]}
\newcommand{\rangoca}[2]{[#1\twodots#2)}
\newcommand{\rangoaa}[2]{(#1\twodots#2)}

%ejercicios
\newtheorem{exercise}{Ejercicio}
\newenvironment{ejercicio}[1][]{\begin{exercise}#1\rm}{\end{exercise} \vspace{0.2cm}}
\newenvironment{items}{\begin{enumerate}[a)]}{\end{enumerate}}
\newenvironment{subitems}{\begin{enumerate}[i)]}{\end{enumerate}}
\newcommand{\sugerencia}[1]{\noindent \textbf{Sugerencia:} #1}

\lstnewenvironment{code}{
    \lstset{% general command to set parameter(s)
        language=C++, basicstyle=\small\ttfamily, keywordstyle=\slshape,
        emph=[1]{tipo,usa}, emphstyle={[1]\sffamily\bfseries},
        morekeywords={tint,forn,forsn},
        basewidth={0.47em,0.40em},
        columns=fixed, fontadjust, resetmargins, xrightmargin=5pt, xleftmargin=15pt,
        flexiblecolumns=false, tabsize=2, breaklines, breakatwhitespace=false, extendedchars=true,
        numbers=left, numberstyle=\tiny, stepnumber=1, numbersep=9pt,
        frame=l, framesep=3pt,
    }
   \csname lst@SetFirstLabel\endcsname}
  {\csname lst@SaveFirstLabel\endcsname}


%tipos basicos
\newcommand{\rea}{\ensuremath{\mathsf{Float}}}
\newcommand{\cha}{\ensuremath{\mathsf{Char}}}
\newcommand{\str}{\ensuremath{\mathsf{String}}}

\newcommand{\mcd}{\mathrm{mcd}}
\newcommand{\prm}[1]{\ensuremath{\mathsf{prm}(#1)}}
\newcommand{\sgd}[1]{\ensuremath{\mathsf{sgd}(#1)}}

\newcommand{\tuple}[2]{\ensuremath{#1 \times #2}}

%listas
\newcommand{\TLista}[1]{\ensuremath{seq \langle #1\rangle}}
\newcommand{\lvacia}{\ensuremath{[\ ]}}
\newcommand{\lv}{\ensuremath{[\ ]}}
\newcommand{\longitud}[1]{\ensuremath{|#1|}}
\newcommand{\cons}[1]{\ensuremath{\mathsf{addFirst}}(#1)}
\newcommand{\indice}[1]{\ensuremath{\mathsf{indice}}(#1)}
\newcommand{\conc}[1]{\ensuremath{\mathsf{concat}}(#1)}
\newcommand{\cab}[1]{\ensuremath{\mathsf{head}}(#1)}
\newcommand{\cola}[1]{\ensuremath{\mathsf{tail}}(#1)}
\newcommand{\sub}[1]{\ensuremath{\mathsf{subseq}}(#1)}
\newcommand{\en}[1]{\ensuremath{\mathsf{en}}(#1)}
\newcommand{\cuenta}[2]{\mathsf{cuenta}\ensuremath{(#1, #2)}}
\newcommand{\suma}[1]{\mathsf{suma}(#1)}
\newcommand{\twodots}{\ensuremath{\mathrm{..}}}
\newcommand{\masmas}{\ensuremath{++}}
\newcommand{\matriz}[1]{\TLista{\TLista{#1}}}

\newcommand{\seqchar}{\TLista{\cha}}


% Acumulador
\newcommand{\acum}[1]{\ensuremath{\mathsf{acum}}(#1)}
\newcommand{\acumselec}[3]{\ensuremath{\mathrm{acum}(#1 |  #2, #3)}}

% \selector{variable}{dominio}
\newcommand{\selector}[2]{#1~\ensuremath{\leftarrow}~#2}
\newcommand{\selec}{\ensuremath{\leftarrow}}

\newcommand{\pred}[3]{%
    {\normalfont\bfseries\ttfamily\noindent pred }%
    {\normalfont\ttfamily #1}%
    \ifthenelse{\equal{#2}{}}{}{\ (#2) }%
    \{%
    \begin{adjustwidth}{+2em}{}
      \ensuremath{#3}
    \end{adjustwidth}
    \}%
    {\normalfont\bfseries\,\par}%
}

\newenvironment{proc}[4][res]{%

  % El parametro 1 (opcional) es el nombre del resultado
  % El parametro 2 es el nombre del problema
  % El parametro 3 son los parametros
  % El parametro 4 es el tipo del resultado
  % Preambulo del ambiente problema
  % Tenemos que definir los comandos requiere, asegura, modifica y aux
  \newcommand{\pre}[2][]{%
    {\normalfont\bfseries\ttfamily Pre}%
    \ifthenelse{\equal{##1}{}}{}{\ {\normalfont\ttfamily ##1} :}\ %
    \{\ensuremath{##2}\}%
    {\normalfont\bfseries\,\par}%
  }
  \newcommand{\post}[2][]{%
    {\normalfont\bfseries\ttfamily Post}%
    \ifthenelse{\equal{##1}{}}{}{\ {\normalfont\ttfamily ##1} :}\
    \{\ensuremath{##2}\}%
    {\normalfont\bfseries\,\par}%
  }
  \renewcommand{\aux}[4]{%
    {\normalfont\bfseries\ttfamily aux\ }%
    {\normalfont\ttfamily ##1}%
    \ifthenelse{\equal{##2}{}}{}{\ (##2)}\ : ##3\, = \ensuremath{##4}%
    {\normalfont\bfseries\,;\par}%
  }
  \renewcommand{\pred}[3]{%
    {\normalfont\bfseries\ttfamily pred }%
    {\normalfont\ttfamily ##1}%
    \ifthenelse{\equal{##2}{}}{}{\ (##2) }%
    \{%
    \begin{adjustwidth}{+5em}{}
      \ensuremath{##3}
    \end{adjustwidth}
    \}%
    {\normalfont\bfseries\,\par}%
  }

  \newcommand{\res}{#1}
  \vspace{1ex}
  \noindent
  \encabezadoDeProc{#1}{#2}{#3}{#4}
  % Abrimos la llave
  \{\par%
  \tocarEspacios
}
% Ahora viene el cierre del ambiente problema
{
  % Cerramos la llave
  \noindent\}
  \vspace{1ex}
}


\newcommand{\aux}[4]{%
    {\normalfont\bfseries\ttfamily\noindent aux\ }%
    {\normalfont\ttfamily #1}%
    \ifthenelse{\equal{#2}{}}{}{\ (#2)}\ : #3\, = \ensuremath{#4}%
    {\normalfont\bfseries\,;\par}%
}


% \newcommand{\pre}[1]{\textsf{pre}\ensuremath{(#1)}}

\newcommand{\procnom}[1]{\textsf{#1}}
\newcommand{\procil}[3]{\textsf{proc #1}\ensuremath{(#2) = #3}}
\newcommand{\procilsinres}[2]{\textsf{proc #1}\ensuremath{(#2)}}
\newcommand{\preil}[2]{\textsf{Pre #1: }\ensuremath{#2}}
\newcommand{\postil}[2]{\textsf{Post #1: }\ensuremath{#2}}
\newcommand{\auxil}[2]{\textsf{fun }\ensuremath{#1 = #2}}
\newcommand{\auxilc}[4]{\textsf{fun }\ensuremath{#1( #2 ): #3 = #4}}
\newcommand{\auxnom}[1]{\textsf{fun }\ensuremath{#1}}
\newcommand{\auxpred}[3]{\textsf{pred }\ensuremath{#1( #2 ) \{ #3 \}}}

\newcommand{\comentario}[1]{{/*\ #1\ */}}

\newcommand{\nom}[1]{\ensuremath{\mathsf{#1}}}


% En las practicas/parciales usamos numeros arabigos para los ejercicios.
% Aca cambiamos los enumerate comunes para que usen letras y numeros
% romanos
\newcommand{\arreglarincisos}{%
  \renewcommand{\theenumi}{\alph{enumi}}
  \renewcommand{\theenumii}{\roman{enumii}}
  \renewcommand{\labelenumi}{\theenumi)}
  \renewcommand{\labelenumii}{\theenumii)}
}



%%%%%%%%%%%%%%%%%%%%%%%%%%%%%% PARCIAL %%%%%%%%%%%%%%%%%%%%%%%%
\let\@xa\expandafter
\newcommand{\tituloparcial}{\centerline{\depto -- \lamateria}
  \centerline{\elnombre -- \lafecha}%
  \setlength{\TPHorizModule}{10mm} % Fija las unidades de textpos
  \setlength{\TPVertModule}{\TPHorizModule} % Fija las unidades de
                                % textpos
  \arreglarincisos
  \newcounter{total}% Este contador va a guardar cuantos incisos hay
                    % en el parcial. Si un ejercicio no tiene incisos,
                    % cuenta como un inciso.
  \newcounter{contgrilla} % Para hacer ciclos
  \newcounter{columnainicial} % Se van a usar para los cline cuando un
  \newcounter{columnafinal}   % ejercicio tenga incisos.
  \newcommand{\primerafila}{}
  \newcommand{\segundafila}{}
  \newcommand{\rayitas}{} % Esto va a guardar los \cline de los
                          % ejercicios con incisos, asi queda mas bonito
  \newcommand{\anchodegrilla}{20} % Es para textpos
  \newcommand{\izquierda}{7} % Estos dos le dicen a textpos donde colocar
  \newcommand{\abajo}{2}     % la grilla
  \newcommand{\anchodecasilla}{0.4cm}
  \setcounter{columnainicial}{1}
  \setcounter{total}{0}
  \newcounter{ejercicio}
  \setcounter{ejercicio}{0}
  \renewenvironment{ejercicio}[1]
  {%
    \stepcounter{ejercicio}\textbf{\noindent Ejercicio \theejercicio. [##1
      puntos]}% Formato
    \renewcommand\@currentlabel{\theejercicio}% Esto es para las
                                % referencias
    \newcommand{\invariante}[2]{%
      {\normalfont\bfseries\ttfamily invariante}%
      \ ####1\hspace{1em}####2%
    }%
    \newcommand{\Proc}[5][result]{
      \encabezadoDeProc{####1}{####2}{####3}{####4}\hspace{1em}####5}%
  }% Aca se termina el principio del ejercicio
  {% Ahora viene el final
    % Esto suma la cantidad de incisos o 1 si no hubo ninguno
    \ifthenelse{\equal{\value{enumi}}{0}}
    {\addtocounter{total}{1}}
    {\addtocounter{total}{\value{enumi}}}
    \ifthenelse{\equal{\value{ejercicio}}{1}}{}
    {
      \g@addto@macro\primerafila{&} % Si no estoy en el primer ej.
      \g@addto@macro\segundafila{&}
    }
    \ifthenelse{\equal{\value{enumi}}{0}}
    {% No tiene incisos
      \g@addto@macro\primerafila{\multicolumn{1}{|c|}}
      \bgroup% avoid overwriting somebody else's value of \tmp@a
      \protected@edef\tmp@a{\theejercicio}% expand as far as we can
      \@xa\g@addto@macro\@xa\primerafila\@xa{\tmp@a}%
      \egroup% restore old value of \tmp@a, effect of \g@addto.. is

      \stepcounter{columnainicial}
    }
    {% Tiene incisos
      % Primero ponemos el encabezado
      \g@addto@macro\primerafila{\multicolumn}% Ahora el numero de items
      \bgroup% avoid overwriting somebody else's value of \tmp@a
      \protected@edef\tmp@a{\arabic{enumi}}% expand as far as we can
      \@xa\g@addto@macro\@xa\primerafila\@xa{\tmp@a}%
      \egroup% restore old value of \tmp@a, effect of \g@addto.. is
      % global
      % Ahora el formato
      \g@addto@macro\primerafila{{|c|}}%
      % Ahora el numero de ejercicio
      \bgroup% avoid overwriting somebody else's value of \tmp@a
      \protected@edef\tmp@a{\theejercicio}% expand as far as we can
      \@xa\g@addto@macro\@xa\primerafila\@xa{\tmp@a}%
      \egroup% restore old value of \tmp@a, effect of \g@addto.. is
      % global
      % Ahora armamos la segunda fila
      \g@addto@macro\segundafila{\multicolumn{1}{|c|}{a}}%
      \setcounter{contgrilla}{1}
      \whiledo{\value{contgrilla}<\value{enumi}}
      {%
        \stepcounter{contgrilla}
        \g@addto@macro\segundafila{&\multicolumn{1}{|c|}}
        \bgroup% avoid overwriting somebody else's value of \tmp@a
        \protected@edef\tmp@a{\alph{contgrilla}}% expand as far as we can
        \@xa\g@addto@macro\@xa\segundafila\@xa{\tmp@a}%
        \egroup% restore old value of \tmp@a, effect of \g@addto.. is
        % global
      }
      % Ahora armo las rayitas
      \setcounter{columnafinal}{\value{columnainicial}}
      \addtocounter{columnafinal}{-1}
      \addtocounter{columnafinal}{\value{enumi}}
      \bgroup% avoid overwriting somebody else's value of \tmp@a
      \protected@edef\tmp@a{\noexpand\cline{%
          \thecolumnainicial-\thecolumnafinal}}%
      \@xa\g@addto@macro\@xa\rayitas\@xa{\tmp@a}%
      \egroup% restore old value of \tmp@a, effect of \g@addto.. is
      \setcounter{columnainicial}{\value{columnafinal}}
      \stepcounter{columnainicial}
    }
    \setcounter{enumi}{0}%
    \vspace{0.2cm}%
  }%
  \newcommand{\tercerafila}{}
  \newcommand{\armartercerafila}{
    \setcounter{contgrilla}{1}
    \whiledo{\value{contgrilla}<\value{total}}
    {\stepcounter{contgrilla}\g@addto@macro\tercerafila{&}}
  }
  \newcommand{\grilla}{%
    \g@addto@macro\primerafila{&\textbf{TOTAL}}
    \g@addto@macro\segundafila{&}
    \g@addto@macro\tercerafila{&}
    \armartercerafila
    \ifthenelse{\equal{\value{total}}{\value{ejercicio}}}
    {% No hubo incisos
      \begin{textblock}{\anchodegrilla}(\izquierda,\abajo)
        \begin{tabular}{|*{\value{total}}{p{\anchodecasilla}|}c|}
          \hline
          \primerafila\\
          \hline
          \tercerafila\\
          \tercerafila\\
          \hline
        \end{tabular}
      \end{textblock}
    }
    {% Hubo incisos
      \begin{textblock}{\anchodegrilla}(\izquierda,\abajo)
        \begin{tabular}{|*{\value{total}}{p{\anchodecasilla}|}c|}
          \hline
          \primerafila\\
          \rayitas
          \segundafila\\
          \hline
          \tercerafila\\
          \tercerafila\\
          \hline
        \end{tabular}
      \end{textblock}
    }
  }%
  % \datosalumno
}

\newcommand{\datosalumno}{
  \vspace{0.4cm}
  \textbf{Apellidos:}

  \textbf{Nombres:}

  \textbf{LU:}

  \textbf{Correo electrónico:}

  \textbf{Nro. de carillas que adjunta:}
  \vspace{0.5cm}
}


% AMBIENTE CONSIGNAS
% Se usa en el TP para ir agregando las cosas que tienen que resolver
% los alumnos.
% Dentro del ambiente hay que usar \item para cada consigna

\newcounter{consigna}
\setcounter{consigna}{0}

\newenvironment{consignas}{%
  \newcommand{\consigna}{\stepcounter{consigna}\textbf{\theconsigna.}}%
  \renewcommand{\ejercicio}[1]{\item ##1 }
  \renewcommand{\proc}[5][result]{\item
    \encabezadoDeProc{##1}{##2}{##3}{##4}\hspace{1em}##5}%
  \newcommand{\invariante}[2]{\item%
    {\normalfont\bfseries\ttfamily invariante}%
    \ ##1\hspace{1em}##2%
  }
  \renewcommand{\aux}[4]{\item%
    {\normalfont\bfseries\ttfamily aux\ }%
    {\normalfont\ttfamily ##1}%
    \ifthenelse{\equal{##2}{}}{}{\ (##2)}\ : ##3 \hspace{1em}##4%
  }
  % Comienza la lista de consignas
  \begin{list}{\consigna}{%
      \setlength{\itemsep}{0.5em}%
      \setlength{\parsep}{0cm}%
    }
}%
{\end{list}}



% para decidir si usar && o ^
\newcommand{\y}[0]{\ensuremath{\land}}

% macros de correctitud
\newcommand{\semanticComment}[2]{#1 \ensuremath{#2};}
\newcommand{\namedSemanticComment}[3]{#1 #2: \ensuremath{#3};}


\newcommand{\local}[1]{\semanticComment{local}{#1}}

\newcommand{\vale}[1]{\semanticComment{vale}{#1}}
\newcommand{\valeN}[2]{\namedSemanticComment{vale}{#1}{#2}}
\newcommand{\impl}[1]{\semanticComment{implica}{#1}}
\newcommand{\implN}[2]{\namedSemanticComment{implica}{#1}{#2}}
\newcommand{\estado}[1]{\semanticComment{estado}{#1}}

\newcommand{\invarianteCN}[2]{\namedSemanticComment{invariante}{#1}{#2}}
\newcommand{\invarianteC}[1]{\semanticComment{invariante}{#1}}
\newcommand{\varianteCN}[2]{\namedSemanticComment{variante}{#1}{#2}}
\newcommand{\varianteC}[1]{\semanticComment{variante}{#1}}

\usepackage{caratula} % Version modificada para usar las macros de algo1 de ~> https://github.com/bcardiff/dc-tex

\begin{document}

\titulo{TP de Especificación}
\subtitulo{Esperando el Bondi}
\fecha{30 de Marzo de 2022}
\materia{Algoritmos y Estructuras de Datos I}
\grupo{Grupo 1}

\newcommand{\dato}{\textit{Dato}}
\newcommand{\individuo}{\textit{Individuo}}

\newcommand{\tiempo}{\textit{Tiempo}}
\newcommand{\dist}{\textit{Dist}}
\newcommand{\gps}{\textit{GPS}}
\newcommand{\recorrido}{\textit{Recorrido}}
\newcommand{\viaje}{\textit{Viaje}}
\newcommand{\nombr}{\textit{Nombre}}
\newcommand{\grilla}{\textit{Grilla}}
\newcommand{\celda}{\textit{Celda}}


% Pongan cuantos integrantes quieran
\integrante{Polonuer, Joaquin}{1612/21}{jtpolonuer@gmail.com}
\integrante{González, Facundo}{1440/21}{facundo2gonzalez2@gmail.com}
\integrante{Jaime, Brian David}{411/18}{brian.d.jaime97@gmail.com}
\integrante{Guberman, Diego}{469/17}{diego98g@hotmail.com}

\maketitle


\section{Definición de Tipos}
\begin{description}
    \item type $\tiempo = \float$
    \item type $\dist = \float$
    \item type $\gps = \float \times \float$
    \item type $\recorrido = \TLista{\gps}$
    \item type $\viaje = \TLista{\tiempo \times \gps}$
    \item type $\nombr = \ent \times \ent$
    \item type $\grilla = \TLista{\gps \times \gps \times \nombr}$
    \item type $\celda = \gps \times \gps \times \nombr$
\end{description}


\section{Constantes}
% \aux{MIN}{}{\ent}{1}
% \aux{MAX}{}{\ent}{10}


\section{Problemas}

\subsection{Ejercicio 1}

Devolver verdadero si los puntos GPS del viaje y los tiempos están en rango.

\begin{proc}{viajeValido}{\In v: \viaje, \Out res: $\bool$}{}
    \pre{True}
    \post{res = \True \leftrightarrow esViajeValido(v)}

    \pred{esViajeValido}{v: \viaje}{
        (\forall i:\ent) (0 \leq i < \longitud{v} \implicaLuego (esTiempoValido(v[i]_0) \y sonCoordenadasValidas(v[i]_1)))
    }

    \pred{esTiempoValido}{t: \tiempo}{
        t \geq 0
    }

    \pred{sonCoordenadasValidas}{c: \gps}{
        -90.0 \leq c_0 \leq 90.0 \y -180.0 \leq c_1 \leq 180.0
    }

\end{proc}

\pagebreak

\subsection{Ejercicio 2}

Devolver verdadero si los puntos GPS del recorrido están en rango.

\begin{proc}{recorridoValido}{\In v: \recorrido, \Out res: $\bool$}{}
    \pre{True}
    \post{res = \True \leftrightarrow esRecorridoValido(v)}

    \pred{esRecorridoValido}{v: \recorrido}{
        (\forall i:\ent) (0 \leq i < \longitud{v} \implicaLuego sonCoordenadasValidas(v[i]))
    }

\end{proc}

\pagebreak

\subsection{Ejercicio 3}

Chequear que todos los puntos registrados en un viaje válido se encuentren dentro de un círculo de radio r kilómetros.

\begin{proc}{enTerritorio}{\In v: \viaje, \In r: \dist, \Out res: $\bool$}{}
    \pre{esViajeValido(v)}
    \post{res = \True \leftrightarrow estaEnTerritorio(v,r)}

    \pred{estaEnTerritorio}{v: \viaje, r: \dist}{
        (\exists c: \gps)(sonCoordenadasValidas(c) \yLuego (\forall i: \ent)(0 \leq i < \longitud{v} \implicaLuego dist(c,v[i]_1) \leq 1000 \cdot r)) \\
        \text{/* Multiplico r por 1000 dado que r está dado en kilómetros y la función auxiliar $dist(p1,p2)$} \\
        \text{devuelve su resultado en metros */}
    }

\end{proc}

\pagebreak

\subsection{Ejercicio 4}

Dado un viaje válido, determinar el tiempo total que tardó el colectivo. Este valor debe ser calculado como el tiempo transcurrido desde el primer punto registrado y hasta el último.

\begin{proc}{tiempoTotal}{\In v: \viaje, \Out t: \tiempo}{}
    \pre{esViajeValido(v)}
    \post{esMaximaDiferenciaTiempo(v,t)}

    \pred{esMaximaDiferenciaTiempo}{v: \viaje, t: \tiempo}{
        \comentario{t es la diferencia entre dos tiempos del v} \\
        (\exists i,j: \ent)(0 \leq i,j < \longitud{v} \yLuego v[i]_0 - v[j]_0 = t) \y \\
        \comentario{t es la mayor diferencia posible entre dos tiempos del viaje} \\
        \neg(\exists n,m: \ent)(0 \leq n,m < \longitud{v} \yLuego v[n]_0 - v[m]_0 > t)
    }

    %Se me ocurrió lo siguiente para este. Qué les parece? - Diego
    %esta copada la idea 
    %igual dejaria la que hablamos en el labo, por simplicidad - Polo
    %Revisando me parece mejor la anterior - Diego
    
    % \pred{maximaDiferenciaTiempo}{v: \viaje, t: \tiempo}{
    %     (\exists ti,t_{f}: \tiempo)((esTiempoValido(ti) \y esTiempoValido(t_{f})) \yLuego (esMinimoTiempo(v, ti) \y \\ esMaximoTiempo(v, t_{f})) \y t = t_{f} - ti)
    % }

    % \pred{esMinimoTiempo}{v: \viaje, t: \tiempo}{
    %     (\exists i: \ent)(0 \leq i < \longitud{v} \yLuego (\forall j: \ent)(0 \leq j < \longitud{v} \implicaLuego v[i]_0 \leq v[j]_0) \y t = v[i]_0)
    % }

    % \pred{esMaximoTiempo}{v: \viaje, t: \tiempo}{
    %     (\exists i: \ent)(0 \leq i < \longitud{v} \yLuego (\forall j: \ent)(0 \leq j < \longitud{v} \implicaLuego v[i]_0 \geq v[j]_0) \y t = v[i]_0)
    % }

\end{proc}
%Ojo que dice que "no necesariamente están ordenados" en las mediciones de los viajes

\pagebreak

\subsection{Ejercicio 5}

Dado un viaje válido, determinar la distancia recorrida en kilómetros aproximada utilizando toda la información registrada en el viaje, es decir, utilizando la información registrada de todos los tramos.

\begin{proc}{distanciaTotal}{\In v: \viaje, \Out d: \dist}{}
    \pre{esViajeValido(v)}
    \post{distanciaViajeOrdenado(v,d)}

    \pred{distanciaViajeOrdenado}{v: \viaje, d: \dist}{
        (\exists v': \viaje)(esElViajeOrdenado(v,v') \y d = sumaDistanciasSucesivas(v'))
    }

    \pred{esElViajeOrdenado}{v,v': \viaje}{
        estaOrdenadoTemporalmente(v') \y esPermutacion(v,v')
    }

    \pred{estaOrdenadoTemporalmente}{v: \viaje}{
        (\forall i:\ent)(0 \leq i < \longitud{v}-1 \implicaLuego v[i]_0 < v[i+1]_0)
    }

    \pred{esPermutacion}{v1,v2: \viaje}{
        (\forall e: \tiempo \times \gps)(apariciones(v1,e) = apariciones (v2,e))
    }

    \aux{apariciones}{v: \viaje, e: $\tiempo \times \gps$}{\ent}{\sum \limits_{i=0}^{\longitud{v}-1} \IfThenElse{v[i] = e}{1}{0}
    }

    \aux{sumaDistanciasSucesivas}{v: \viaje}{\dist}{
    \frac{1}{1000} \cdot \sum\limits_{i=0}^{\longitud{v}-2} dist(v[i]_1,v[i+1]_1) \\
    \text{/* Divido la sumatoria por 1000 dado que se pide el resultado en kilómetros y la función auxiliar} \\
    \text{$dist(p1,p2)$ devuelve su resultado en metros */}
    }

\end{proc}

% \subsection{Ejercicio 5, Alternativa}

% \begin{proc}{distanciaTotal}{\In v: \viaje, \Out d: \dist}{}
%     \pre{esViajeValido(v)}
%     \post{esDistanciaTotal(v,d)}

%     \pred{esDistanciaTotal}{v: \viaje, d: \dist}{
%          (\exists v': \viaje)(esPermutacion(v,v') \y estaOrdenadoTemporalmente(v') \y d = sumaDistanciasSucesivas(v'))
%      }

%      \pred{estaOrdenadoTemporalmente}{v: \viaje}{
%          (\forall i:\ent)(0 \leq i < \longitud{v}-1 \implicaLuego v[i]_0 < v[i+1]_0)
%      }

%     \pred{esPermutacion}{v1,v2: \viaje}{
%     (\forall e: \tiempo \times \gps)(\# apariciones(v1,e) = \# apariciones (v2,e))
%      }

%      \aux{apariciones}{v: \viaje, e: $\tiempo \times \gps$}{\ent}{\sum\limits_{i=0}^{\longitud{v}-1} \IfThenElse{v[i] = e}{1}{0}
%      }

%     \aux{sumaDistanciasSucesivas}{v: \viaje}{\dist}{
%     \frac{1}{1000} \cdot \sum\limits_{i=0}^{\longitud{v}-2} dist(v[i]_1,v[i+1]_1)}

% \end{proc}

\pagebreak

\subsection{Ejercicio 6}

Dado un viaje válido devolver verdadero si el colectivo superó los 80 km/h en algún momento del viaje.

\begin{proc}{excesoDeVelocidad}{\In v: \viaje, \Out res: $\bool$}{}
    \pre{esViajeValido(v)}
    \post{res = \True \leftrightarrow superaVelocidad(v)}

    \pred{superaVelocidad}{v: \viaje}{
        (\exists i,j: \ent)(0 \leq i,j < \longitud{v} \yLuego esTramo(v, v[i],v[j]) \y velocidadTramo(v[i],v[j]) > 80)
    }
    
    \pred{esTramo}{v: \viaje, e1,e2: $\tiempo \times \gps$}{
        e1_0 < e2_0 \y \neg(\exists e: \tiempo \times \gps)(e \in v \y e1_0 < e_0 < e2_0)
    }
    
    \aux{velocidadTramo}{e1,e2 : $\tiempo \times \gps$}{\float}{
    \frac{dist(e1_1,e2_1)}{e2_0 - e1_0} \cdot 3.6 \\
    \text{/* Multiplico por 3,6 dado que se pide el resultado en kilómetros por hora y la función auxiliar} \\
    \text{$dist(p1,p2)$ devuelve su resultado en metros mientras que los tiempos están en segundos */}
    }
    
\end{proc}

% Hice otra forma donde usamos predicados que hicimos en el ejercicio 5 - Diego

\begin{proc}{excesoDeVelocidad}{\In v: \viaje, \Out res: $\bool$}{}
    \pre{esViajeValido(v)}
    \post{res = \True \leftrightarrow superaVelocidad(v)}

    \pred{superaVelocidad}{v: \viaje}{
        (\exists v': \viaje)(esElViajeOrdenado(v,v') \y viajeOrdenadoSuperaVelocidad(v'))
    }
    
    \pred{viajeOrdenadoSuperaVelocidad}{v: \viaje}{
        (\exists i: \ent)(0 \leq i < \longitud{v}-1 \yLuego velocidadTramo(v[i],v[i+1]) > 80)
    }
    
    \aux{velocidadTramo}{e1,e2 : $\tiempo \times \gps$}{\float}{
    \frac{dist(e1_1,e2_1)}{e2_0 - e1_0} \cdot 3.6 \\
    \text{/* Multiplico por 3,6 dado que se pide el resultado en kilómetros por hora y la función auxiliar} \\
    \text{$dist(p1,p2)$ devuelve su resultado en metros mientras que los tiempos están en segundos */}
    }
    
\end{proc}

\pagebreak

\subsection{Ejercicio 7}

Dada una lista de viajes válidos, calcular la cantidad de viajes que se encontraban en ruta en cualquier momento entre $\mathrm{t_0}$ y $\mathrm{t_f}$ inclusives. Por ejemplo, si un viaje comenzó a las 13:30 y terminó a las 14:30 y la franja es de 14:00 a 15:00, el viaje debería estar considerado. Lo mismo ocurre si el viaje comenzó a las 14:10 y terminó a las 14:15 o si comenzó a las 13:30 y terminó a las 16:00.

\begin{proc}{flota}{\In vs: \TLista{\viaje}, \In $\mathrm{t_0}$: \tiempo, \In $\mathrm{t_f}$: \tiempo, \Out res: \ent}{}
    \pre{sonTodosViajesValidos(vs) \y t_0 \leq t_f \y esTiempoValido(t_0) \y esTiempoValido(t_f)}
    \post{esCantidadEnRuta(vs, t_0, t_f, res)}
   
    \pred{sonTodosViajesValidos}{vs: \TLista{\viaje}}{
        (\forall v:\viaje)(v \in vs \implicaLuego esViajeValido(v))
    }

    \pred{esCantidadEnRuta}{vs: \TLista{\viaje}, $\mathrm{t_0}$,$\mathrm{t_f}$: \tiempo, res: \ent}{
        \existe{vs'}{\TLista{\viaje}}{
            \paraTodo{v}{\viaje}{
                (v \in vs \y estaEnRuta(v, t_0, t_f))
                \implicaLuego
                \#apariciones(v, vs') = \#apariciones(v, vs)
            }
            \y \longitud{vs'} = res
        }
    }
    
    \pred{estaEnRuta}{v: \viaje, $\mathrm{t_0}$,$\mathrm{t_f}$: \tiempo}{
        \existe{i, j}{\ent}{
            0 \leq i, j < \longitud{v} 
            \yLuego
            v[i]_0 \leq t_0 < t_{f} \leq v[j]_0
        }
        \vee
        \existe{i}{\ent}{
            0 \leq i < \longitud{v} 
            \yLuego
            t_0 \leq v[i]_0 \leq t_{f}
        }
    }
    
    \pred{estaEnRuta}{v: \viaje, $\mathrm{t_0}$,$\mathrm{t_f}$: \tiempo}{
        (\exists i,j: \ent)(0 \leq i \leq j < \longitud{v} \yLuego (v[i]_0 \leq t_f \y v[j]_0 \geq t_0))
    }
    
\end{proc}

\pagebreak
 
\subsection{Ejercicio 8}

Dado un viaje v válido, un recorrido r válido y un umbral u (en kilómetros), devolver todos los puntos del recorrido que no fueron
cubiertos por ningún punto del viaje. Se considera que un punto p del recorrido está cubierto si al menos un punto del viaje
está a menos de u kilómetros del punto p.

\begin{proc}{recorridoCubierto}{\In v: \viaje, \In r: \recorrido, \In u \dist, \Out res: \TLista{\gps}}{}
    \pre{esViajeValido(v) \y u > 0 \y esRecorridoValido(r)}
    \post{sonTodosLosPuntosNoCubiertos(res, v, r, u)}
    
    \pred{sonTodosLosPuntosNoCubiertos}{res: \TLista{\gps}, v: \viaje, r: \recorrido, u: \dist}{
        \text{/*Todos los puntos que están en res son puntos no cubiertos del recorrido*/}\\
        \text{/*Todos los puntos no cubiertos del recorrido están en res*/}\\
        \paraTodo{p}{\gps}{
            p \in res 
            \leftrightarrow
            (p \in r \y \neg{estaCubierto(p,v,u)})
        }
    }
    
    \pred{estaCubierto}{p: \gps, v: \viaje, u: \dist}{
        \existe{m}{\tiempo \times \gps}{
            m \in v \yLuego dist(m_1, p) < u \cdot 1000
        }
    }

\end{proc}

\pagebreak

\subsection{Ejercicio 9}

Que dados dos puntos GPS, construye una grilla de n × m. 

\begin{proc}{construirGrilla}{\In esq1: \gps, \In esq2: \gps, \In n: \ent, \In m: \ent, \Out g: \grilla}{}{
    \pre{sonEsquinasValidas(esq1, esq2) \y n > 0 \y m > 0}
    \post{esGrillaCorrecta(esq1, esq2, n, m, g)}
    
    \pred{sonEsquinasValidas}{esq1,esq2: \gps}{
        sonCoordenadasValidas(esq1) \y sonCoordenadasValidas(esq2) \y esq1_0 > esq2_0 \y esq1_1 < esq2_1
    }
    
    \pred{esGrillaCorrecta}{esq1,esq2: \gps, n,m: \ent, g: \grilla}{
        \longitud{g} = m \cdot n \y
        esquinasSonCombLineales(esq1, esq2, n, m, g)
    }
    
    % \pred{infDerechaCorrecta}{g: \grilla}{
    % \text{/*Toda coordenada inferior derecha tiene que ser 
    %         la inferior izquierda más el tamaño de la celda */}\\
    %     \paraTodo{i}{\ent}{
    %         0 \leq i < \longitud{g} \\
    %         \implicaLuego 
    %         esqInfDer(g[i]) = \\ 
    %         (esqSupDer(g[i])_0 - tamanoCelda(esq1, esq2, n, m)_0, \\ esqSupDer(g[i])_0 + tamanoCelda(esq1, esq2, n, m)_1)
    %     }    
    % }
    
    \pred{esquinasSonCombLineales}{esq1,esq2: \gps, n,m: \ent, g: \grilla}{
        \paraTodo{a,b}{\ent}{
           (1 \leq a \leq n \y 1 \leq b \leq m) \implicaLuego
           \existe{i}{\ent}{
                0 \leq i < \longitud{g} \yLuego \\
                \text{/* Esquina superior izquierda*/} \\
                g[i]_0 = esqSupIzq(a,b,n,m,esq1,esq2) \y \\
                \text{/* Esquina inferior derecha */} \\
                g[i]_1 = esqInfDer(a,b,n,m,esq1,esq2) \y \\
                \text{/* Nombre */} \\
                g[i]_2 = nombre(a,b)
            }
        }    
    }

    \aux{esqSupIzq}{a,b,n,m: \ent, esq1,esq2: \gps}{\gps}{
    (esq1_0 - (a-1) \cdot (tamanoCelda(esq1, esq2, n, m))_0,\\ esq1_1 + (b-1) \cdot tamanoCelda(esq1, esq2, n, m)_1)
    }
    
    \aux{esqInfDer}{a,b,n,m: \ent, esq1,esq2: \gps}{\gps}{
    (esq1_0 - a \cdot (tamanoCelda(esq1, esq2, n, m))_0,\\ esq1_1 + b \cdot tamanoCelda(esq1, esq2, n, m)_1)
    }
    
    \aux{nombre}{a,b: \ent}{\nombr}{(a,b)}

    \aux{tamanoCelda}{esq1,esq2: \gps, n,m:\ent}{$\float \times \float$}{(\frac{esq1_0 - esq2_0}{n},\frac{esq2_1 - esq1_1}{m})}

}

\end{proc}

\pagebreak

\subsection{Ejercicio 10}

Que dado un recorrido, devuelve la secuencia ordenada de regiones visitadas por el colectivo.

\begin{proc}{regiones}{\In r: \recorrido, \In g: \grilla, \Out res: \TLista{\nombr}}{}

\pre{esRecorridoValido(r) \y esGrillaDelRecorrido(g,r)}
    \post{esSecuenciaDelRecorrido(res,r,g)}
    
    \pred{esSecuenciaDelRecorrido}{res: \TLista{\nombr}, r: \recorrido, g:\grilla}{
    \longitud{res} = \longitud{r} \y
    \paraTodo{i}{\ent}{
    0 \leq i < \longitud{res}
    \implicaLuego
    \existe{c}{\celda}{c \in g \yLuego (nombre((c_2)_0,(c_2)_1)=res[i] \y estaEnCelda(r[i],c))
    }
    }
}
    \pred{estaEnCelda}{gps: \gps, c: \celda}{
    (((c)_1)_0 \leq (gps)_0 \leq ((c)_0)_0) \y
    (((c)_0)_1 \leq (gps)_1 \leq ((c)_1)_1)
}

    \pred{esGrillaDelRecorrido}{g: \grilla, r: \recorrido}{
    \paraTodo{i}{\ent}{
    0 \leq i < \longitud{r} \implicaLuego
    \existe{c}{\celda}{c \in g \yLuego estaEnCelda(r[i],c)}
    }
    
    
}


    
    
    
\end{proc}


\pagebreak

\subsection{Ejercicio 11}

Que dado un viaje valido y una grilla, determine cuantos saltos hay en el viaje

\begin{proc}{cantidadDeSaltos}{\In g: \grilla, \In v: \viaje, \Out res: \TLista{\ent}}{}{
    \pre{esViajeValido(v) \wedge esGrillaDelViaje(g, v)}
    \post{esCantidadDeSaltos(g,v,res)}
    \pred{esCantidadDeSaltos}{g: \grilla, v: \viaje, res: \ent}{
        \existe{v'}{\viaje}{
            esElViajeOrdenado(v', v)
            \wedge \existe{R}{\TLista{\nombr}}{
                esSecuenciaDelViaje(R, v', g)
                \wedge cantidadDeSaltos(R) = res
            }
        }
    }
    \aux{cantidadDeSaltos}{R: \TLista{\nombr}}{\ent}{\\
        \sum \limits_{i=0}^{\longitud{v}-1} \IfThenElse{esCeldaContigua(R[i], R[i+1])}{0}{1}
    }
    
    \aux {esCeldaContigua}{n1: \nombr, n2: \nombr}{\bool}{\\
        \IfThenElse{
            (\longitud{n1_0 - n2_0} \leq 1
            \wedge 
            \longitud{n1_1 - n2_1} \leq 1)
        }{true}{false}
    }
    
    \pred{esGrillaDelViaje}{g: \grilla, v: \viaje}{
    \paraTodo{i}{\ent}{
    0 \leq i < \longitud{v} \implicaLuego
    \existe{c}{\celda}{c \in g \yLuego estaEnCelda((v[i])_1,c)}
    }
    
}
}
    
\end{proc}


\pagebreak

\subsection{Ejercicio 12}

Se cuenta con un viaje valido de mas de 5 puntos, y la lista errores que indica cada momento para el cual el valor registrado por el GPS fue erroneo y que debe ser corregido automaticamente

\begin{proc}{corregirViaje}{\Inout v: \viaje, \In errores: \TLista{\tiempo}}{}{
    \pre{\longitud{v} > 5
        \wedge esViajeValido(v)
        \wedge sonTiemposValidos(errores) \\
        \wedge 10 * \longitud{errores} < \longitud{v} 
        \wedge primeroYUltimoSinErrores(v, errores) 
        \wedge v = v_0
    }
    
    \post{esViajeCorregido(v, v_0, errores)}
    
    \pred{sonTiemposValidos}{e: \TLista{\tiempo}}{
    \paraTodo{i}{\ent}{
        0\leq i < |e| \implicaLuego esTiempoValido(i)
        }
    }
    
    \pred{esViajeCorregido}{}{
        
    }
    
}
    
\end{proc}
\pagebreak

\subsection{Ejercicio 13}
Que dada una lista de viajes válidos, calcule el histograma de velocidades máximas registradas entre todos los viajes.

\begin{proc}{histograma}{\In xs: \TLista{\viaje}, \In bins: \ent, \Out cuentas: \TLista{\ent}, \Out limites: \TLista{\float}}{}{
\pre{sonViajesValidos(xs) \y bins>0}
\post{sonCuentasCorrectas(xs,bins,limites) \y sonLimitesCorrectos(limites, xs, bins)}

\pred{sonCuentasCorrectas}{xs: \TLista{\viaje}, bins: \ent, limites: \TLista{\float}}{
\longitud{cuentas}=bins \y \\
\existe{vels}{\TLista{\float}}{
sonVelocidadesMaximas(vels,xs) \y \\
(\paraTodo{i}{\ent}{(0 \leq i < \longitud{cuentas}-1) \implicaLuego cuentas[i] = cantEnIntervalo(vels, limites[i], limites[i+1]))} \y \\
(cuentas[\longitud{cuentas}-1]=cantEnIntervaloCerrado(vels,limites[\longitud{cuentas}-1],limites[\longitud{cuentas}]))
} 
}
\pred{sonLimitesCorrectos}{limites: \TLista{\float}, xs: \TLista{\viaje}, bins: \ent}{
\longitud{limites}=bins+1 \y 
estaOrdenado(limites) \y \\
\existe{vels}{\TLista{\float}}{
sonVelocidadesMaximas(vels,xs)  \y  \\
\paraTodo{i}{\ent}{ 
(0 \leq i < \longitud{limites}) \implicaLuego \existe{min, max}{\float}{(esMinimo(vels, min) \y esMaximo(vels, max)) \y limites[i] = min + (i \times \frac{max-min}{bins})}
}}
}

\pred{sonVelocidadesMaximas}{vels: \TLista{\float}, xs: \TLista{\viaje}}{

}

\aux{cantEnIntervalo}{vels: \TLista{\float}, lim1: \float, lim2: \float}{\float}{

}

\aux{cantEnIntervaloCerrado}{vels: \TLista{\float}, lim1: \float, lim2: \float}{\float}{

}

\pred{esMinimo}{vels: \TLista{\float}, min: \float}{
min \in vels \y \paraTodo{i}{\ent}{
0 \leq i < \longitud{vels} \implicaLuego min \leq vels[i]}
}
\pred{esMaximo}{vels: \TLista{\float}, max: \float}{
max \in vels \y \paraTodo{i}{\ent}{
0 \leq i < \longitud{vels} \implicaLuego vels[i] \leq max}
}

}


\end{proc}

\end{document}

